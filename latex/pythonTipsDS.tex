%% Generated by Sphinx.
\def\sphinxdocclass{report}
\documentclass[letterpaper,12pt,english]{sphinxmanual}
\ifdefined\pdfpxdimen
   \let\sphinxpxdimen\pdfpxdimen\else\newdimen\sphinxpxdimen
\fi \sphinxpxdimen=.75bp\relax

\usepackage[utf8]{inputenc}
\ifdefined\DeclareUnicodeCharacter
 \ifdefined\DeclareUnicodeCharacterAsOptional\else
  \DeclareUnicodeCharacter{00A0}{\nobreakspace}
\fi\fi
\usepackage{cmap}
\usepackage[T1]{fontenc}
\usepackage{amsmath,amssymb,amstext}
\usepackage{babel}
\usepackage{times}
\usepackage[Bjarne]{fncychap}
\usepackage{longtable}
\usepackage{sphinx}

\usepackage{geometry}
\usepackage{multirow}
\usepackage{eqparbox}

% Include hyperref last.
\usepackage{hyperref}
% Fix anchor placement for figures with captions.
\usepackage{hypcap}% it must be loaded after hyperref.
% Set up styles of URL: it should be placed after hyperref.
\urlstyle{same}

\addto\captionsenglish{\renewcommand{\figurename}{Fig.}}
\addto\captionsenglish{\renewcommand{\tablename}{Table}}
\addto\captionsenglish{\renewcommand{\literalblockname}{Listing}}

\addto\extrasenglish{\def\pageautorefname{page}}

\setcounter{tocdepth}{1}

\usepackage{amsmath}
\usepackage{mathtools}
\usepackage{amsfonts}
\usepackage{amssymb}
\usepackage{dsfont}
\def\Z{\mathbb{Z}}
\def\R{\mathbb{R}}
\def\bX{\mathbf{X}}
\def\X{\mathbf{X}}
\def\By{\mathbf{y}}
\def\Bbeta{\boldsymbol{\beta}}
\def\bU{\mathbf{U}}
\def\bV{\mathbf{V}}
\def\V1{\mathds{1}}
\def\hU{\mathbf{\hat{U}}}
\def\hS{\mathbf{\hat{\Sigma}}}
\def\hV{\mathbf{\hat{V}}}
\def\E{\mathbf{E}}
\def\F{\mathbf{F}}
\def\x{\mathbf{x}}
\def\h{\mathbf{h}}
\def\v{\mathbf{v}}
\def\nv{\mathbf{v^{{f -}}}}
\def\nh{\mathbf{h^{{f -}}}}
\def\s{\mathbf{s}}
\def\b{\mathbf{b}}
\def\c{\mathbf{c}}
\def\W{\mathbf{W}}
\def\C{\mathbf{C}}
\def\P{\mathbf{P}}
\def\T{{\bf \mathcal T}}
\def\B{{\bf \mathcal B}}
\def\euler{\ e^{i\pi} + 1 = 0}


\title{Python Tips for Data Scientist}
\date{February 22, 2019}
\release{1.00}
\author{Wenqiang Feng}
\newcommand{\sphinxlogo}{\sphinxincludegraphics{logo.png}\par}
\renewcommand{\releasename}{Release}
\makeindex

\begin{document}

\maketitle
\sphinxtableofcontents
\phantomsection\label{\detokenize{index::doc}}\phantomsection\label{\detokenize{index:index}}\begin{quote}
\phantomsection\label{\detokenize{index:fig-logo}}\begin{figure}[htbp]
\centering

\noindent\sphinxincludegraphics{{logo}.png}
\label{\detokenize{index:fig-logo}}\end{figure}
\end{quote}

Welcome to my \sphinxstylestrong{Python Tips for Data Scientist} notes! In those notes, you will learn some useful tips for Data Scientist daily work. The PDF version can be downloaded from \sphinxhref{pythonTipsDS.pdf}{HERE}.




\chapter{Preface}
\label{\detokenize{preface:id1}}\label{\detokenize{preface::doc}}\label{\detokenize{preface:contents}}\label{\detokenize{preface:preface}}
\begin{sphinxadmonition}{note}{Chinese proverb}

The palest ink is better than the best memory. \textendash{} old Chinese proverb
\end{sphinxadmonition}


\section{About}
\label{\detokenize{preface:about}}

\subsection{About this tutorial}
\label{\detokenize{preface:about-this-tutorial}}
This document is a summary of my valueable experiences in using \sphinxcode{Python} for \sphinxcode{Data Scientist} daily work. The PDF version can be downloaded from \sphinxhref{sphinxgithub.pdf}{HERE}. \sphinxstylestrong{You may download and distribute it. Please be aware, however, that the note contains typos as well as inaccurate or incorrect description.}

In this repository, I try to use the detailed \sphinxcode{Data Scientist} related demo code and
examples to share some useful python tips for Data Scientist work. If you find your work wasn’t cited in this note, please feel free to let me know.

Although I am by no means a python programming and Data Scientist expert,
I decided that it would be useful for me to share what I learned
about Python in the form of easy tutorials with detailed example.
I hope those tutorials will be a valuable tool for your studies.

The tutorials assume that the reader has a preliminary knowledge of \sphinxcode{python} programing, \sphinxcode{LaTex} and \sphinxcode{Linux}. And this document is generated automatically by using \sphinxhref{http://sphinx.pocoo.org}{sphinx}.


\subsection{About the authors}
\label{\detokenize{preface:sphinx}}\label{\detokenize{preface:about-the-authors}}\begin{itemize}
\item {} 
\sphinxstylestrong{Wenqiang Feng}
\begin{itemize}
\item {} 
Data Scientist and PhD in Mathematics

\item {} 
University of Tennessee at Knoxville

\item {} 
Email: \sphinxhref{mailto:von198@gmail.com}{von198@gmail.com}

\end{itemize}

\item {} 
\sphinxstylestrong{Biography}

Wenqiang Feng is Data Scientist within DST’s Applied Analytics Group. Dr. Feng’s responsibilities include providing DST clients with access to cutting-edge skills and technologies, including Big Data analytic solutions, advanced analytic and data enhancement techniques and modeling.

Dr. Feng has deep analytic expertise in data mining, analytic systems, machine learning algorithms, business intelligence, and applying Big Data tools to strategically solve industry problems in a cross-functional business. Before joining DST, Dr. Feng was an IMA Data Science Fellow at The Institute for Mathematics and its Applications (IMA) at the University of Minnesota. While there, he helped startup companies make marketing decisions based on deep predictive analytics.

Dr. Feng graduated from University of Tennessee, Knoxville, with Ph.D. in Computational Mathematics and Master’s degree in Statistics. He also holds Master’s degree in Computational Mathematics from Missouri University of Science and Technology (MST) and Master’s degree in Applied Mathematics from the University of Science and Technology of China (USTC).

\item {} 
\sphinxstylestrong{Declaration}

The work of Wenqiang Feng was supported by the IMA, while working at IMA. However, any opinion, finding, and conclusions or recommendations expressed in this material are those of the author and do not necessarily reflect the views of the IMA, UTK and DST.

\end{itemize}


\section{Motivation for this tutorial}
\label{\detokenize{preface:motivation-for-this-tutorial}}
No matter you like it or not, Python has been one of the most popular programming languages.
I have been using \sphinxcode{Python} for almost 4 years. Frankly speaking, I wasn't impressed and attracted
by \sphinxcode{Python} at the first using. After starting working in industry, I have to use \sphinxcode{Python}. Graduately
I recognize the elegance of Python and use it as one of my main programming language. But I foud that:
\begin{itemize}
\item {} 
Most of the \sphinxcode{Python} books or tutorials which emphasize on programming will overwhelme the green hand.

\item {} 
While most of the \sphinxcode{Python} books or tutorials \sphinxcode{Data Scientist} or \sphinxcode{Data Analysis} didn't cover some essential skills from the engineer side.

\end{itemize}

So I want to keep some of my valuable tips which are heavily applied in my daily work.


\section{Feedback and suggestions}
\label{\detokenize{preface:feedback-and-suggestions}}
Your comments and suggestions are highly appreciated. I am more than happy to receive
corrections, suggestions or feedbacks through email (Wenqiang Feng: \sphinxhref{mailto:von198@gmail.com}{von198@gmail.com}) for improvements.


\chapter{Main Reference}
\label{\detokenize{reference:main-reference}}\label{\detokenize{reference::doc}}\label{\detokenize{reference:reference}}
\begin{sphinxthebibliography}{VanderPlas2016}
\bibitem[VanderPlas2016]{\detokenize{VanderPlas2016}}{\phantomsection\label{\detokenize{reference:vanderplas2016}} 
Jake VanderPlas. \sphinxhref{https://tanthiamhuat.files.wordpress.com/2018/04/pythondatasciencehandbook.pdf}{Python Data Science Handbook: Essential Tools for Working with Data, 2016.}
}
\bibitem[McKinney2013]{\detokenize{McKinney2013}}{\phantomsection\label{\detokenize{reference:mckinney2013}} 
Wes McKinney. \sphinxhref{http://bedford-computing.co.uk/learning/wp-content/uploads/2015/10/Python-for-Data-Analysis.pdf}{Python for Data Analysis, 2013.}
}
\bibitem[Georg2018]{\detokenize{Georg2018}}{\phantomsection\label{\detokenize{reference:georg2018}} 
Georg Brandl. \sphinxhref{https://media.readthedocs.org/pdf/sphinx/1.7/sphinx.pdf}{Sphinx Documentation, Release 1.7.10+, 2018.}
}
\end{sphinxthebibliography}



\renewcommand{\indexname}{Index}
\printindex
\end{document}